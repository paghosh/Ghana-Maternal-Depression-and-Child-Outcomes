\documentclass[12pt]{article}

% Page layout
\usepackage[margin=1in, headheight=15pt]{geometry}

% Tables
\usepackage{booktabs}
\usepackage{tabularx}
\usepackage{multirow}
\usepackage{array}

% Fonts and encoding
\usepackage[T1]{fontenc}
\usepackage[utf8]{inputenc}
\usepackage{mathpazo}  % Palatino font for professional look

% Math
\usepackage{amsmath}

% Spacing
\usepackage{setspace}
\onehalfspacing

% Headers
\usepackage{fancyhdr}
\pagestyle{fancy}
\fancyhf{}
\rhead{\thepage}
\lhead{\footnotesize Ghosh (2026) --- Prenatal Depression and Child Cognitive Development}
\renewcommand{\headrulewidth}{0.4pt}

% Hyperlinks
\usepackage[colorlinks=true,linkcolor=blue,citecolor=blue]{hyperref}

% Table and figure numbering
\usepackage{caption}
\captionsetup{labelfont=bf, font=small}

% Allow float placement
\usepackage{float}
\usepackage{graphicx}  % for \resizebox

% Title
\title{\textbf{Prenatal Depression and Child Cognitive Development in Ghana} \\[0.5em]
\large Tables and Results}
\author{Pallab Ghosh \\
\textit{University of Oklahoma}}
\date{February 2026 \\ Draft 1}

\begin{document}

\maketitle
\thispagestyle{empty}

\vfill
\noindent\textbf{Data Source:} Ghana Socioeconomic Panel Survey (GSPS), Waves 1--3 (2009, 2014, 2018). \\
\noindent\textbf{Note:} Prenatal depression identified via two strategies: (A) mothers pregnant at the time of K10 depression measurement, and (B) retrospective birth timing linking children born 0--9 months after a previous wave's interview to the mother's K10 score at that wave. All regression tables report standard errors clustered at the enumeration area (EA) level in parentheses. Significance levels: *** $p<0.01$, ** $p<0.05$, * $p<0.10$.

\clearpage
\tableofcontents
\clearpage

%===============================================================================
%   MAIN RESULTS
%===============================================================================
\section{Main Results}

%--- Table P1 ---
\begin{table}[htbp]\centering
\def\sym#1{\ifmmode^{#1}\else\(^{#1}\)\fi}
\caption{Effect of Prenatal Depression on Child Cognitive Development}
\begin{tabular}{l*{5}{c}}
\toprule
            &\multicolumn{1}{c}{(1)}&\multicolumn{1}{c}{(2)}&\multicolumn{1}{c}{(3)}&\multicolumn{1}{c}{(4)}&\multicolumn{1}{c}{(5)}\\
            &\multicolumn{1}{c}{OLS}&\multicolumn{1}{c}{OLS}&\multicolumn{1}{c}{EA FE}&\multicolumn{1}{c}{EA FE}&\multicolumn{1}{c}{EA FE}\\
\midrule
Prenatal depression (std. K10)&      -0.007         &      -0.009         &      -0.002         &                     &      -0.027         \\
            &     (0.042)         &     (0.038)         &     (0.052)         &                     &     (0.057)         \\
\addlinespace
Prenatally depressed (K10 $\geq$ 30)&                     &                     &                     &       0.184         &                     \\
            &                     &                     &                     &     (0.263)         &                     \\
\addlinespace
Concurrent depression (std. K10)&                     &                     &                     &                     &       0.035         \\
            &                     &                     &                     &                     &     (0.053)         \\
\midrule
Controls    &          No         &         Yes         &         Yes         &         Yes         &         Yes         \\
EA FE       &          No         &          No         &         Yes         &         Yes         &         Yes         \\
Wave FE     &          No         &          No         &         Yes         &         Yes         &         Yes         \\
Observations&         779         &         779         &         724         &         724         &         724         \\
$R^2$       &       0.000         &       0.071         &       0.297         &       0.298         &       0.297         \\
\bottomrule
\multicolumn{6}{l}{\footnotesize Standard errors clustered at the EA level in parentheses.}\\
\multicolumn{6}{l}{\footnotesize Prenatal depression identified via (A) pregnancy at interview and}\\
\multicolumn{6}{l}{\footnotesize (B) birth timing (child born 0--9 months after previous wave interview).}\\
\multicolumn{6}{l}{\footnotesize Controls: child age, child gender, mother's age,}\\
\multicolumn{6}{l}{\footnotesize household size, and log per capita consumption.}\\
\multicolumn{6}{l}{\footnotesize *** p$<$0.01, ** p$<$0.05, * p$<$0.1}\\
\end{tabular}
\end{table}


\clearpage

%--- Table P2 ---
\begin{table}[htbp]\centering
\def\sym#1{\ifmmode^{#1}\else\(^{#1}\)\fi}
\caption{Prenatal Depression: Sensitivity Analyses}
\setlength{\tabcolsep}{2.8pt}
\small
\begin{tabular}{l*{6}{c}}
\toprule
       &{(1)}&{(2)}&{(3)}&{(4)}&{(5)}&{(6)}\\
       &{Baseline}&{6-month}&{12-month}&{Raven's}&{+Current}&{Strat.\ A}\\
\midrule
Prenatal dep.\ (std.\ K10)& $-$0.002 &       &       & $-$0.003 & $-$0.027 & 0.049 \\
       & (0.052)&       &       & (0.057)& (0.057)& (0.068)\\
\addlinespace
Prenatal dep.\ (6-mo)&       & $-$0.049 &       &       &       &       \\
       &       & (0.056)&       &       &       &       \\
\addlinespace
Prenatal dep.\ (12-mo)&       &       & $-$0.027 &       &       &       \\
       &       &       & (0.039)&       &       &       \\
\addlinespace
Concurrent dep.\ (std.\ K10)&       &       &       &       & 0.035 &       \\
       &       &       &       &       & (0.053)&       \\
\midrule
Timing window& 0--9 mo& 0--6 mo& 0--12 mo& 0--9 mo& 0--9 mo& Strat.\ A only\\
Outcome& Cog Index& Cog Index& Cog Index& Raven's& Cog Index& Cog Index\\
Observations& 724  & 33   & 104  & 724  & 724  & 590  \\
$R^2$  & 0.297& 0.952& 0.738& 0.262& 0.297& 0.292\\
\bottomrule
\multicolumn{7}{l}{\footnotesize Standard errors clustered at the EA level in parentheses.}\\
\multicolumn{7}{l}{\footnotesize All models include EA and wave fixed effects plus full controls.}\\
\multicolumn{7}{l}{\footnotesize *** p$<$0.01, ** p$<$0.05, * p$<$0.1}\\
\end{tabular}
\end{table}



\end{document}
